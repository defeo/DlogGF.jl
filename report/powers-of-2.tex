\documentclass[a4paper,11pt]{article}
\usepackage[utf8]{inputenc}
\usepackage[T1]{fontenc}
\usepackage[english]{babel}
\usepackage{amsmath,amssymb,amsthm,amsopn}
\usepackage{mathrsfs}
\usepackage{graphicx}
%\usepackage{tikz}
%\usepackage{array}
%\usepackage[top=1cm,bottom=1cm]{geometry}
%\usepackage{listings}
%\usepackage{xcolor}
\usepackage{hyperref}
\hypersetup{
    colorlinks=true,
    linkcolor=blue,
    citecolor=red,
}

\newtheoremstyle{break}%
{}{}%
{\itshape}{}%
{\bfseries}{}%  % Note that final punctuation is omitted.
{\newline}{}

\newtheoremstyle{sc}%
{}{}%
{}{}%
{\scshape}{}%  % Note that final punctuation is omitted.
{\newline}{}

\theoremstyle{break}
\newtheorem{thm}{Theorem}[section]
\newtheorem{lm}[thm]{Lemma}
\newtheorem{prop}[thm]{Proposition}
\newtheorem{cor}[thm]{Corollary}

\theoremstyle{sc}
\newtheorem{exo}{Exercise}

\theoremstyle{definition}
\newtheorem{defi}[thm]{Definition}
\newtheorem{ex}[thm]{Example}

\theoremstyle{remark}
\newtheorem{rem}[thm]{Remark}

\DeclareMathOperator{\Ker}{Ker}
\DeclareMathOperator{\Id}{Id}
\DeclareMathOperator{\Img}{Im}
\DeclareMathOperator{\Card}{Card}
\DeclareMathOperator{\Vect}{Vect}
\DeclareMathOperator{\Tr}{Tr}


% Nouvelles commandes
\newcommand{\ps}[2]{\left\langle#1,#2\right\rangle}
\newcommand{\ent}[2]{[\![#1,#2]\!]}
\newcommand{\diff}{\mathop{}\!\mathrm{d}}
\newcommand{\ie}{\emph{i.e. }}

% opening
\title{}
\author{The powers-of-2 descent}


\begin{document}

\maketitle

%\begin{abstract}

%\end{abstract}

%\tableofcontents

%\clearpage

\section{Introduction}

The discrete logarithm problem has seen dramatic improvements these last
years in the small characteristic case~\cite{Joux13, BGJT13, GKZ14},
culminating with two \emph{quasi-polynomial} algorithms in 2014. ``On the powers
of $2$'' is the paper~\cite{GKZ14} introducing one of those two algorithms and this document is
meant to give some explanations about it. Before going into details about this
algorithm, we recall a few facts about discrete logarithm. A more complete
panorama can be found in suveys such as~\cite{JP16, GKZ16}.

\section{The Discrete Logarithm Problem (DLP)}
Let $G=\left\langle g\right\rangle$ be a cyclic group generated by an element
$g$, and denote by $N=|G|$ its cardinal. We have the isomorphism:
\[
 \begin{array}{cccc}
   exp_g: & \mathbb{Z}/N\mathbb{Z} & \to & G \\
   & n & \mapsto & g^n,
 \end{array}
\]
and we denote by $\log_g=\exp_g^{-1}$ the inverse isomorphism. In practice, the
\emph{square and multiply} algorithm allows us to compute $g^n=\exp_g(n)$
efficiently, \ie in polynomial time. But, given $y = g^k$, the computation of $k
= \log_g(y)$ is not as easy. This kind of function $f$, where $f$ is easy to
compute but $f^{-1}$ is hard to compute, are called \emph{one-way} functions. They are
typically used in cryptology to make the encryption fast and the deciphering
slow. The DLP first appeared in the article of Diffie and
Hellman in 1976, ``New Directions in Cryptography''~\cite{DH76}. It was supposed to be a very hard problem,
\ie only an exponential time algorithm was known at that time, and all the
security of the protocol invented in this article relied on the hardness of the
DLP. This article attracted a lot of attention, and so did the DLP, which has
been widely studied since then. Originally, the group involved in the Diffie-Hellman
protocol was $(\mathbb{Z}/N\mathbb{Z})^\times$, but there are other interesting
groups, such as the points of an elliptic curve or the invertible element of a
finite field. Over the years, two types of algorithms emerged:
\begin{itemize}
  \item the \emph{generic} algorithms, that can be used for any group, with an
    exponential complexity of type $O(\sqrt N)$, where $N=|G|$ is the cardinal
    of the group;
  \item the \emph{index calculus} algorithms, that are built on the structure of
    the group considered, and that have very different complexities, depending
    on the kind of the studied group.
\end{itemize}

The recent algorithms that achieve quasi-polynomial complexity are index
calculus algorithms, so we will introduce these algorithms a bit more.

\section{The index calculus method}

Here again, we consider a group $G=\left\langle g \right\rangle$, an element
$g^k = y\in G$, and we want to compute $k$. The index calculus algorithms always
follow the same pattern:
\begin{enumerate}
  \item[0.] we choose $F\subset G$ such that $\left\langle F \right\rangle = G$
    (we often have $g\in F$), this subset is called the \emph{factor base};
  \item we generate multiplicative relations between the elements of $F$, \ie we
    find $f_1, \dots, f_n \in F$ and $e_1, \dots, e_n\in \mathbb{Z}$ such that
    $\prod_i f_i^{e_i} = 1$, that is equivalent to the linear equation $\sum_i
    e_i\log_g(f_i) = 0$;
  \item we solve the linear system with unknowns $\log_g(f_i)$ arising from step
    1 to obtain $\log(f)$ for all $f\in F$;
  \item we find a multiplicative relation between $y$ and elements of $F$, or
    equivalently, we express $\log_g y = k$ as a linear combination of the
  $\log_g(f_i)$.
\end{enumerate}
\textcolor{red}{Example : Hellman-Reyneri algorithm, like in the slides for
  F.Paulin ?}
  \newline
  \newline
  An \textcolor{red}{(other ?)} example is given by the algorithm of Granger,
  Kleinjung and Zumbrägel in \cite{GKZ14}, that we call powers-of-2 algorithm here.
\section{The powers-of-2 algorithm}

Let $G = (\mathbb{F}_{q})^\times$ the group of invertible elements of a finite
field of small characteristic $\mathbb{F}_q$, where $q=p^n$ is a prime
power. 

\clearpage
\bibliographystyle{plain}
\bibliography{dlog}
\end{document}
